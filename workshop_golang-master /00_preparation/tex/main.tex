\documentclass[11pt, letterpaper]{article}
\usepackage[utf8]{inputenc}
\usepackage{hyperref}
\usepackage{enumerate}
\usepackage{listings}
\usepackage{fancyvrb}
\usepackage{kotex}
\usepackage{polski}
\usepackage[document]{ragged2e} % keep all left
\usepackage[english]{babel}

\usepackage{minted} % yaml syntax highlighting

\newenvironment{markdown}%
    {\VerbatimEnvironment\begin{VerbatimOut}{tmp.markdown}}%
    {\end{VerbatimOut}%
        \immediate\write18{pandoc tmp.markdown -t latex -o tmp.tex}%
        \input{tmp.tex}}

\providecommand{\tightlist}{%
  \setlength{\itemsep}{0pt}\setlength{\parskip}{0pt}}

\newcommand*{\email}[1]{\href{mailto:#1}{\nolinkurl{#1}} } 

\title{Golang Programming Workshop\\Preparation\\{ \small \href{https://creativecommons.org/licenses/by/4.0/}{CC BY 4.0} }  }
\author{Wojciech Barczynski}
\date{}


\begin{document}
\selectlanguage{english}

\begin{titlepage}
\maketitle
\end{titlepage}

\tableofcontents
\pagebreak
\section{Prerequisites}

Expected background knowledge and skills for the workshop:

\begin{itemize}%
\item Have 1-year hands-on experience in other programming language.%
\item Know how to work with the Command Line Interface in Linux or {\small OSX}.
\end{itemize}%

\section{Your workstation}

\begin{itemize}%
\item Linux or {\small OSX} recommended;%
\item Basic: \begin{itemize}%
    \item Golang,
    \item Git.
    \end{itemize}%
\item An IDE or code editor to work with Golang, e.g.: \begin{itemize}%
    \item vscode,
    \item Jetbrains Goland.
    \end{itemize}%
\item Docker;
\item {\small SQL} and no{\small SQL} exercise (recommended with Docker):
\begin{itemize}%
    \item Postgres,
    \item MongoDB.
\end{itemize}%
\end{itemize}%

\section{How to prepare your workstation}

\subsection{Ubuntu Linux}

We recommend Ubuntu, one of the LTSes - \href{https://wiki.ubuntu.com/Releases}{wiki.ubuntu.com/Releases}.
\bigskip

1. Install Golang, following the instructions from \href{https://github.com/golang/go/wiki/Ubuntu}{github.com/golang/go/wiki/Ubuntu}:

\begin{minted}{bash}
sudo add-apt-repository ppa:longsleep/golang-backports
sudo apt-get update
sudo apt-get install golang-go
\end{minted}

\begin{minted}{bash}
# check whether it works.
go version
\end{minted}
\bigskip

2. Install vscode (\href{https://code.visualstudio.com/}{code.visualstudio.com/}) or Goland (\href{https://www.jetbrains.com/go/}{jetbrains.com/go/}):

\begin{minted}{bash}
# vscode with snap
# https://code.visualstudio.com/docs/setup/linux
sudo snap install --classic code
\end{minted}

For Jetbrain Goland follow the standalone installation:\\
\href{https://www.jetbrains.com/help/go/installation-guide.html#c2dfc2}{www.jetbrains.com/help/go/installation-guide.html}.

\smallskip
You will find on \href{https://github.com/golang/go/wiki/IDEsAndTextEditorPlugins}{github.com/golang/go/wiki/IDEsAndTextEditorPlugins} information on to configure your favorite editor to develop in Golang.

\bigskip
3. Install Git:

\begin{minted}{bash}
sudo apt-get update
sudo apt-get install git
\end{minted}

\bigskip
4. Install Docker after \href{https://docs.docker.com/engine/install/ubuntu/#install-using-the-repository}{docs.docker.com/engine/install/ubuntu}, copy and paste to your terminal:

\begin{minted}[breaklines]{bash}
# install necessary packages
sudo apt-get update
sudo apt-get install -y ca-certificates curl gnupg

# Add Docker’s official GPG key:
sudo install -m 0755 -d /etc/apt/keyrings
curl -fsSL https://download.docker.com/linux/ubuntu/gpg \
    | sudo gpg --dearmor -o /etc/apt/keyrings/docker.gpg

# Setting up the repository:
echo \
 "deb [arch="$(dpkg --print-architecture)" signed-by=/etc/apt/keyrings/docker.gpg] https://download.docker.com/linux/ubuntu \
 "$(. /etc/os-release && echo "$VERSION_CODENAME")" stable" |\
 sudo tee /etc/apt/sources.list.d/docker.list > /dev/null
\end{minted}

Install \mintinline{bash}{docker-ce}:

\begin{minted}{bash}
sudo apt-get update
sudo apt-get install docker-ce docker-ce-cli
\end{minted}

Check whether it works:

\begin{minted}{bash}
sudo docker run hello-world
\end{minted}

\subsection{MacOS}

1. Install \emph{homebrew}, a package manager for MacOS, follow the instructions from the official website - \href{https://brew.sh/}{brew.sh/}.
\bigskip

2. With \emph{homebrew}, install Golang is easy:

\begin{minted}{bash}
brew install golang
\end{minted}

\begin{minted}{bash}
go version
\end{minted}

\bigskip
3. Choose your {\small IDE} - \emph{vscode} (recommended for this workshop) or \emph{Goland}:

\begin{minted}{bash}
# for vscode
brew install --cask visual-studio-code
\end{minted}

\begin{minted}{bash}
# for goland community edition
brew install --cask goland
\end{minted}

\bigskip
3. To install Docker, go to \href{https://docs.docker.com/desktop/install/mac-install/}{docs.docker.com/desktop/install/mac-install/}.

\section{References}

\begin{itemize}%
\item The Missing Semester of Your CS Education: \href{https://missing.csail.mit.edu/}{missing.csail.mit.edu};
\item Quick start with \emph{Goland}: \href{https://www.jetbrains.com/help/idea/quick-start-guide-goland.html}{jetbrains.com/help/idea/quick-start-guide-goland.html};
\item Quick start with VSCode: \\ \href{https://learn.microsoft.com/en-us/azure/developer/go/configure-visual-studio-code}{learn.microsoft.com/en-us/azure/developer/go/configure-visual-studio-code}.
\end{itemize}%

\end{document}